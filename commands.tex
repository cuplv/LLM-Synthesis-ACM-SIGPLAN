% Comment commands used by Evan. Contain boolean flags to either enable/disable all coments, or enable/disable TODOs and jedi statements.
\newboolean{showcomments}
\setboolean{showcomments}{true}
\newboolean{showjedi}
\setboolean{showjedi}{true}
\newboolean{showtodos}
\setboolean{showtodos}{false}
\newcommand{\editcomment}[2][red]{\ifthenelse{\boolean{showcomments}}{{\color{#1}$\bigl[$\bgroup\em #2\egroup$\bigr]$}}{}}
\newcommand{\note}[1]{\editcomment[blue]{#1}}
\newcommand{\todo}[1]{\ifthenelse{\boolean{showtodos}}{\editcomment{{\bf TODO:} #1}}}
% Evan likes to review "jedi" statements, that should capture the essence of a paragraph (i.e. how would Yoda convey the point of a paragraph?). Should be able to just read the jedi statements to get all the important points of a paper.
\newcommand{\jedi}[1]{\ifthenelse{\boolean{showjedi}}{\editcomment[cyan]{#1}}{}}

\newcommand{\ie}{\textit{i.e.}~}
\newcommand{\eg}{\textit{e.g.}~}

% Note(klinvill): \xspace tries adds a space after uses of `\lang` when
% reasonable. The package's creator recommends not using xspace
% (https://tex.stackexchange.com/questions/86565/drawbacks-of-xspace/86620#86620),
% but I believe that the LaTeX behavior of eating spaces will be more surprising
% to co-authors than the occassional weird spacing introduced by xspace.
% Language Name
%---------------
\newcommand{\lang}{TSQL\xspace}

\newcommand{\bigstep}{\Downarrow}
% Formatting
%---------
\newcommand{\kw}[1]{\texttt{#1}} %keywords
\newcommand{\code}[1]{{\tt #1}}
\newcommand{\C}[1]{\code{#1}}
\newcommand{\tuplee}[1]{\langle #1 \rangle}
\newcommand*{\rom}[1]{\expandafter\romannumeral #1}
\newcommand{\spc}[0]{\quad}
\newcommand{\ALT}{~\mid~}
\newcommand{\ALTE}{\spc\mid\spc}
\newcommand{\conj}{~\wedge~}
\newcommand{\disj}{~\vee~}
\newcommand{\Iv}{I_{\mathrm{le}}^v}
\newcommand{\cle}{\mvc_{\mathrm{le}}}
\newcommand{\blue}[1]{\textcolor{Blue}{#1}}
\newcommand{\brown}[1]{\textcolor{Brown}{#1}}


% Math env
%-----------
\newenvironment{nop}{}{}
\newenvironment{smathpar}{
\begin{nop}\small\begin{mathpar}}{
\end{mathpar}\end{nop}\ignorespacesafterend}

% Relational algebra notation for use in denotational semantics
%--------------
\newcommand{\denote}[1]{\llbracket #1 \rrbracket}
\newcommand{\semof}{\denote}
\newcommand{\projectOp}{\pi}
\newcommand{\project}[2]{\projectOp_{#1}(#2)}
\newcommand{\filterOp}{\sigma}
\newcommand{\filter}[2]{\filterOp_{#1}(#2)}
\newcommand{\joinOp}{\bowtie}
\newcommand{\join}[3]{#1 \bowtie_{#3} #2}
\newcommand{\aggfun}{\mathcal{G}}
\newcommand{\agg}[2]{|#2|_{#1}}
\newcommand{\orderOp}{\omega}
\newcommand{\orderby}[2]{\orderOp_{#1}(#2)}
\newcommand{\groupby}[2]{|#2|_{#1}}
\newcommand{\renameOp}{\rho}
\newcommand{\rename}[2]{\renameOp_{#1}(#2)}
\newcommand{\bind}{\gg=}


\newcommand{\concatRows}{\mathbin{++}}
\newcommand{\concatCols}{\oplus}

\newcommand{\db}{\mathcal{D}}
\newcommand{\denotedb}[1]{\denote{#1}_{\db}}

% Inference rule notation
\newcommand{\RULE}[3]{\inferrule*[Right=#1]{#2}{#3}}
\newcommand{\rulelabel}[1]{\textrm{\sc {#1}}}

\newcommand{\hole}{\square}
\newcommand{\collection}[1]{\overline{#1}}

\newcommand{\cstr}{\mathcal{C}}
\newcommand{\cols}{\sf cols}
\newcommand{\fresh}{\sf fresh}

% Custom commands
\newcommand{\lm}{\mathcal{L}}
\theoremstyle{definition}
\newtheorem{exmp}{Example}[section]
\newtheorem{thrm}{Theorem}[section]
\newtheorem{lma}{Lemma}[section]
